Jonathan prefers Julia to Matlab.

\subsubsection{Julia resources}
Jonathan learned Julia by starting with the \href{https://lectures.quantecon.org/jl/index_learning_julia.html}{QuantEcon introduction to Julia}.

Here are some helpful resources:
\begin{enumerate}
\item \href{https://julialang.org}{julialang.org}: for installation and general info about julia: blogs, publications, conference
\item \href{https://docs.julialang.org/en/stable/manual/introduction/}{docs.julialang.org/en/stable/manual/introduction}: excellent manual for julia
\item \href{https://lectures.quantecon.org/jl/}{lectures.quantecon.org/jl}: an excellent manual for Julia with a macro vibe
\item \href{https://github.com/bkamins/Julia-DataFrames-Tutorial}{github.com/bkamins/Julia-DataFrames-Tutorial}: excellent tutorial on how to read/use DataFrames in Julia
\item \href{https://www.juliabloggers.com/}{juliabloggers.com}: to keep up with advancements in Julia
\item \href{http://www.johnmyleswhite.com/}{johnmyleswhite.com}: for interesting discussions about performance in Julia 
\end{enumerate}

List of very useful packages:
\begin{itemize}
\item \href{https://github.com/JuliaData/DataFrames.jl}{DataFrames.jl}: DataFrames in Julia
\item \href{https://github.com/davidanthoff/ReadStat.jl}{ReadStat.jl}: read \texttt{dta} files in Julia
\item \href{https://github.com/JuliaIO/JLD.jl}{JLD.jl}: fantastic way of storing output in Julia
\item \href{https://github.com/JuliaNLSolvers/Optim.jl}{Optim.jl}: General optimization in julia
\item \href{https://github.com/johnmyleswhite/Calculus.jl}{Calculus.jl}: for automatic diffentiation in Julia (very practical to check your analytical expressions of gradients and hessians)
\end{itemize}

\subsubsection{Notes on using Julia on UChicago's Midway computing cluster}
\begin{itemize}
\item
You cannot connect to the external internet from Midway's compute nodes. 
Julia's packages are pulled from Github, so you must install packages in Julia (and in Stata) when running on a login node.
\item
You can \texttt{Pkg.activate(".")} on a compute node, but you cannot \texttt{Pkg.instantiate()} on a compute node.
You need to install packages, define the \texttt{Project.toml}, \texttt{Pkg.instantiate()} and precompile (\texttt{using} commands once) on a login node before moving to the compute nodes.  
\end{itemize}

\subsubsection{Notes on managing package dependencies in Julia code}
Running Julia code will involve packages.
Here's how we manage them in our repositories.
\begin{itemize}
\item \href{https://julialang.github.io/Pkg.jl/v1/toml-files/}{\texttt{Project.toml} and \texttt{Manifest.toml}}
\begin{itemize}
	\item These two files are central to \texttt{Pkg}, Julia's builtin package manager.
	They make it possible to instantiate the exact same package environment on different machines.
	\item \texttt{Project.toml} describes the project on a high level. The package dependencies and compatibility constraints are listed in the project file.
	\item \texttt{Manifest.toml} is an absolute record of the state of the packages in the environment. This is not pleasant to read.
\end{itemize}
\item In each repository, we create a task called \texttt{setup\_environment} that defines all package dependencies using the \texttt{Project.toml} and \texttt{Manifest.toml} files.
\item In a Julia script, we set the active environment using the following commands:
\begin{lstlisting}
import Pkg
Pkg.activate("../input/Project.toml")
using package_name
\end{lstlisting}
While we only explicitly name \texttt{Project.toml} as an input in this script,
this presumes that the corresponding \texttt{Manifest.toml} file is available in the same directory.
\item In each Makefile that executes a Julia script,
we provide these \texttt{toml} files as inputs by creating symbolic links to \texttt{setup\_environment}:
\begin{lstlisting}[language=make]
../input/Project.toml: ../../setup_environment/output/Project.toml | ../input/Manifest.toml ../input
	ln -s $< $@
../input/Manifest.toml: ../../setup_environment/output/Manifest.toml | ../input
	ln -s $< $@
\end{lstlisting}
Note that we define \texttt{Manifest.toml} as a pre-requisite for \texttt{Project.toml}.
If a list of pre-requisities is generated by parsing the Julia script for all files that start with \texttt{../input/},
only \texttt{Project.toml} will be flagged.
Making \texttt{Manifest.toml} a pre-requisite of \texttt{Project.toml} ensures that both files appear in the \texttt{input} folder.
\end{itemize}
